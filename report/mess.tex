One motivation for looking for philosophy in New Zealand newspapers comes from reflection on the shortcomings of previous attempts to tell the story of philosophy in New Zealand. The history of philosophy in New Zealand tends to be the history of philosophy as a university discipline. This leaves the historian of philosophy without much to say until the middle years of the twentieth century. The two-volume History of Philosophy in Australia and New Zealand, really begins in earnest around the late 1920s. Some early New Zealand figures are mentioned, notably the Otago professors Duncan MacGregor (Professor of Moral and Mental Philosophy from 1871-1886), and William Salmond (Professor of Moral and Mental Philosophy from 1887-1913). However, it is noted that ‘many of those who had longstanding chairs published next to nothing’ and that the Australasian (and especially New Zealand) philosophical community was small and not well connected \cite[24]{davies-2014}. This leaves someone applying the usual methods of history of philosophy, with their focus on scholarly publications and correspondence, without much to say.

However, we need not look merely at scholarly publications. Newspapers of the time contained reports of the activities of these philosophers, and correspondence and editorial material concerning their thought.\footnote{Consider, for instance, the newspaper debates over Salmond's short publication \textit{The Reign of Grace}: [ADD LINKS].} Newspapers thus represent a possible route for gaining greater insight into the history of academic philosophy in New Zealand. But, moreover, they offer an opportunity for expanding the targets of historical investigation. That is, expanding our conception of what might count as philosophical writing, to include material in newspapers, might allow us to also expand our conception of who counts as a philosopher or as someone engaged in philosophical inquiry.

There are two further advantages to extending our attention beyond scholarly publications towards newspapers. First, as Oddie and Perrett note, philosophy in New Zealand has been determined by northern hemisphere intellectual programmes. They present the flourishing of philosophical programmes unique to, or characteristic of, New Zealand as something to be hoped for in the future \cite[ix--x]{oddie-1992}. % Add pages to the bibtex record.
It is possible that expanding the range of people whose thought we consider might increase the ‘New Zealandness’ of the material we find. Second, the demographic range of early academics in New Zealand philosophy is incredibly small. Indeed, almost all of the early philosophy academics in New Zealand were Scottish men. It is quite possible that attention to philosophical writing in newspapers will, for instance, allow us to find philosophical writing by women as part of the story of philosophical reflection in New Zealand.\footnote{This expansion would be even more impressive if we were able to incorporate writing from the Māori niupepa tradition, however this will not be possible within the current project.}

As it happens, there are many interesting connections between newspapers and philosophy in Australasia. One of the first newspapers in New Zealand, \textit{The New Zealand Advertiser and Bay of Islands Gazette}, was produced by Barzillai Quaife a Congregationalist minister who went on to be the first professional philosophy teacher in Australia and to publish \textit{The Intellectual Sciences}, a book which has a claim to being the first philosophical monograph produced in Australasia \cite[16--17]{davies-2014}. This example is singular, but further encourages the idea that early intellectual life, and philosophy in particular, may be tracable through early New Zealand newspapers.

The consideration of newspaper material also leaves an opening for the introduction of data science methods. Filtering millions of articles, almost all of which are not interesting to the historian of philosophy, to find the relevant material is a difficult information retrieval problem. Simple keyword searching, say for the word 'philosophy', will miss many relevant articles and include many irrelevant ones. This motivates the attempt to use more advanced data science techniques to create a corpus of articles relevant to the historian of phliosophy in New Zealand.

Moreover, once such a corpus has been constructed, data science methods will
again be relevant. One of the core ideas of corpus analysis is `distant reading'. Whereas traditional humanities approaches rely on a `close reading' of particular texts, perhaps writing whole articles or monographs on the meaning of a paticular word in a particular sentence, corpus analysis aims to make claims about collections of texts which are too large for any one scholar to read. [cite idea of distant reading]

Methods of this sort have been applied to questions in the history of philosophy in the past. [Weatherson, Davies, other corpus analysis material]. Most of these studies have focused on academic writing in philosophy. This project extends this work by changing the focus to philosophy carried out in newspapers. It is hoped that this extension will provide new insights into how philosophical ideas have been taken up in newspapers and how newspaper philosophical discourse might have its own interests and concepts not found in academic philosophy.

Similarly, newspaper content has been the focus of research in the digital humanities. [Some examples].



---


The first question is straight-forwardly motivated by the discussion above. Given a very large dataset with a very small subset of interest, we will need to find some method to pick out the small subset. For this question, the main criterion of success will be whether the resulting corpus contains material which is of interest to the historian of philosophy. % We might also be interested in the question of completeness. That is, we might wonder how many articles of interest are missed by our methods.

% ADD SOMETHING FOR DESIRE TO CAPTURE LETTERS, LECTURE REPORTS, and FIRST-ORDER PHILOSOPHY.

The second and third questions are designed to be easily investigable given the material expected to be in the dataset. The period covered by the dataset, from the 1840s up to the turn of the 20th century, is one in which the theory of evolution was developed and propogated. Given this, we expect that there will be discourse concerning the relationship between religion and the sciences.\footnote{The classic anthologies on this relationship in the history of science are [NUMBERS citations]}
This will count as philosophical discourse in so far as it concerns the evaluation of metaphysical and epistemological claims from two different areas of human experience. One area in which we might expect insight into philosophy in early New Zealand newspapers is if we can pick out material which argues for conflict and material which argues for compatability between religion and the sciences.

The third question touches on the motivation to see if the range of discussions in the history of New Zealand philosophy might include material which is not merely following European and North American discussions. The question will be answered positively if the methods deployed in the project can find evidence of application of Western philosophical ideas to a new context, perhaps in terms of ideas of land ownership, relationships with Māori, or debate over the development of new colonial institutions.

The second and third questions do not exhaust the motivations discussed in the previous section. Rather, they are intended to provide two example of the kind of insight which can be obtained from the corpus extracted in the first part of the project.

---

For instance, the METS file for the first issue of \textit{The Press} for 1861 (25/05/1861) contains a dmdSec or ‘descriptive metadata’ section for both the original print version and its electronic counterpart; for the issue as a whole, one for each article (15 in this case, each with the title of the article (no bylines in this case); and an empty section entitled ‘SECTION1’. There is an amdSec, or ‘administrative metadata’ section for the scan of each page. This contains information about the digitisation process and the properties of the resulting image files. A fileGrp section contains links to the scanned images and the ALTO files for each page. Note that the scanned images are not part of the data release. A single \ttt{<structMap LABEL=”Physical Structure”>} section is used to associate the images with page numbers and an order of pages.  A \ttt{<structMap LABEL=”Logical Structure”>} section then provides information about each article in the issue, linking it to a ocr file and area. Areas have names such as ‘P6\_CB00004’.

The ALTO XML file for the first page of the first issue of \texttt{The Press} contains three sections (according to the veridian page this is standard): a \ttt{<Description>}, \ttt{<Styles>}, and \ttt{<Layout>} element. The description element contains information about the file and its sources. For instance, the image it was taken from, the software version used to preprocess the image and create the OCR. In this case, FineReader 7.1 was used. It also tells us that the measurement unit is millimeters. The \ttt{<styles>} Element contains the fonts, sizes and styles used on the page. For instance, Times New Roman in bold is given the id attirbute ‘TXT\_8’ we also have a series of paragraph alignment styles. Finally, the layout element contains the content, location, and dimensions of each text string in the given measurements. It is divided into TextBlock elements with TextLine children. These TextLines in turn contain String elements, such as ‘\ttt{<String ID="P1\_ST00024" HPOS="336" VPOS="733" WIDTH="104" HEIGHT="35" CONTENT="apology" WC="1.00" CC="1110011"/>}’.

---

In such cases, it was decided that they would only be labelled as philosophy if the majority of the article was dedicated to philosophical discourse.
